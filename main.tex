%This is my super simple Real Analysis Homework template

\documentclass{article}
\usepackage[utf8]{inputenc}
\usepackage[english]{babel}
\usepackage[]{amsthm} %lets us use \begin{proof}
\usepackage[]{amssymb} %gives us the character \varnothing
\usepackage{enumitem}
\usepackage{amsmath}
\usepackage{xunicode}
\title{Homework on SVD }
\author{Le Tan Dang Khoa}
\date\today
\begin{document}
\maketitle 
\section*{Problem set 6.2.1:}
\subsection*{Solve}
Consider that two matrices $A=\left[\begin{array}{ll}1 & 2 \\ 0 & 3\end{array}\right]$ and $A=\left[\begin{array}{ll}1 & 1 \\ 3 & 3\end{array}\right]$

The objective is to find that factor these matrices into $A=X \Lambda X^{-1}$

Here, $X$ is eigenvector matrix, $\Lambda$ is eigenvalue matrix.

Find the eigenvalues of the matrix $A=\left[\begin{array}{ll}1 & 2 \\ 0 & 3\end{array}\right]$

For that, consider $\operatorname{det}(A-\lambda I)=0$

$$
\begin{aligned}
|A-\lambda I| & =\left[\begin{array}{ll}
1 & 2 \\
0 & 3
\end{array}\right]-\lambda\left[\begin{array}{ll}
1 & 0 \\
0 & 1
\end{array}\right] \\
& =\left[\begin{array}{cc}
1-\lambda & 2 \\
0 & 3-\lambda
\end{array}\right] \\
& =(1-\lambda)(3-\lambda)-0 \\
& =3-\lambda-3 \lambda+\lambda^{2}
\end{aligned}
$$

Hence,

$$
\begin{aligned}
\operatorname{det}(A-\lambda I) & =0 \\
\lambda^{2}-\lambda-3 \lambda+3 & =0 \\
(\lambda-1)(\lambda-3) & =0
\end{aligned}
$$

Therefore, the eigenvalues of the matrix $A=\left[\begin{array}{ll}1 & 2 \\ 0 & 3\end{array}\right]$ are $\lambda=1, \lambda=3$

Find the corresponding eigenvectors for the eigenvalues $\lambda=1, \lambda=3$

The eigenvector for $\lambda=1$ is as follows:

$$
\begin{aligned}
(A-\lambda I) x & =0 \\
(A-I) x & =0 \\
\left(\left[\begin{array}{ll}
1 & 2 \\
0 & 3
\end{array}\right]-\left[\begin{array}{ll}
1 & 0 \\
0 & 1
\end{array}\right]\right) x & =0 \\
{\left[\begin{array}{ll}
0 & 2 \\
0 & 2
\end{array}\right] x } & =0
\end{aligned}
$$

$\left[\begin{array}{ll}0 & 2 \\ 0 & 2\end{array}\right]\left[\begin{array}{l}x_{1} \\ x_{2}\end{array}\right]=0$

Here,

$2 x_{2}=0$

$x_{2}=0$

And $x_{1}$ is a free variable,

Take $x_{1}=k$

$\left[\begin{array}{l}k \\ 0\end{array}\right]=k\left[\begin{array}{l}1 \\ 0\end{array}\right]$

Therefore, the corresponding eigenvector is $\left[\begin{array}{l}1 \\ 0\end{array}\right]$.

The eigenvector for $\lambda=3$ is as follows:

$$
\begin{aligned}
(A-\lambda I) x & =0 \\
(A-3 I) x & =0 \\
\left(\left[\begin{array}{ll}
1 & 2 \\
0 & 3
\end{array}\right]-\left[\begin{array}{ll}
3 & 0 \\
0 & 3
\end{array}\right]\right) x & =0 \\
{\left[\begin{array}{cc}
-2 & 2 \\
0 & 0
\end{array}\right]\left[\begin{array}{l}
x_{1} \\
x_{2}
\end{array}\right] } & =\left[\begin{array}{l}
0 \\
0
\end{array}\right]
\end{aligned}
$$

Therefore, from equation (2) obtain, $-2 x_{1}+2 x_{2}=0$

Hence, the corresponding eigenvector is $\left[\begin{array}{l}1 \\ 1\end{array}\right]$.

Hence, the eigenvector matrix is $X=\left[\begin{array}{ll}1 & 1 \\ 0 & 1\end{array}\right]$

And inverse of the matrix $X$ is $X^{-1}=\left[\begin{array}{cc}1 & -1 \\ 0 & 1\end{array}\right]$

The eigenvalue matrix is $\Lambda=\left[\begin{array}{ll}1 & 0 \\ 0 & 3\end{array}\right]$
Therefore, the matrix $A$ can be written as $A=\left[\begin{array}{ll}1 & 1 \\ 0 & 1\end{array}\right]\left[\begin{array}{ll}1 & 0 \\ 0 & 3\end{array}\right]\left[\begin{array}{cc}1 & -1 \\ 0 & 1\end{array}\right]$

Find the eigenvalues of the matrix $A=\left[\begin{array}{ll}1 & 1 \\ 3 & 3\end{array}\right]$

For that, consider $\operatorname{det}(A-\lambda I)=0$
$$
\begin{aligned}
|A-\lambda I| & =\left[\begin{array}{ll}
1 & 1 \\
3 & 3
\end{array}\right]-\lambda\left[\begin{array}{ll}
1 & 0 \\
0 & 1
\end{array}\right] \\
& =\left[\begin{array}{cc}
1-\lambda & 1 \\
3 & 3-\lambda
\end{array}\right] \\
& =(1-\lambda)(3-\lambda)-3 \\
& =3-\lambda-3 \lambda+\lambda^{2}-3
\end{aligned}
$$

$=\lambda^{2}-\lambda-3 \lambda$

$=\lambda(\lambda-4)$

Hence,

$$
\begin{aligned}
\operatorname{det}(A-\lambda I) & =0 \\
\lambda(\lambda-4) & =0
\end{aligned}
$$

Therefore, the eigenvalues of the matrix $A=\left[\begin{array}{ll}1 & 2 \\ 0 & 3\end{array}\right]$ are $\lambda=0, \lambda=4$

Find the corresponding eigenvectors for the eigenvalues $\lambda=0, \lambda=4$

The eigenvector for $\lambda=0$ is as follows:

$$
\begin{array}{r}
(A-\lambda I) x=0 \\
(A-0 I) x=0 \\
{\left[\begin{array}{ll}
1 & 1 \\
3 & 3
\end{array}\right]\left[\begin{array}{l}
x_{1} \\
x_{2}
\end{array}\right]=0}
\end{array}
$$

Therefore,

$$
\begin{array}{r}
x_{1}+x_{1}=0 \\
3 x_{1}+3 x_{1}=0
\end{array}
$$

Hence, the corresponding eigenvector is $\left[\begin{array}{c}-1 \\ 1\end{array}\right]$.

The eigenvector for $\lambda=4$ is as follows:

$$
\begin{aligned}
(A-\lambda I) x & =0 \\
(A-4 I) x & =0 \\
\left(\left[\begin{array}{ll}
1 & 1 \\
3 & 3
\end{array}\right]-\left[\begin{array}{ll}
4 & 0 \\
0 & 4
\end{array}\right]\right) x & =0 \\
{\left[\begin{array}{cc}
-3 & 1 \\
3 & -1
\end{array}\right]\left[\begin{array}{l}
x_{1} \\
x_{2}
\end{array}\right] } & =\left[\begin{array}{l}
0 \\
0
\end{array}\right]
\end{aligned}
$$

Therefore, from equation (2) obtain,

$$
\begin{aligned}
-3 x_{1}+x_{2} & =0 \\
3 x_{1}-x_{2} & =0
\end{aligned}
$$

Hence, the corresponding eigenvector is $\left[\begin{array}{l}1 \\ 3\end{array}\right]$.

Hence, the eigenvector matrix is $X=\left[\begin{array}{cc}-1 & 1 \\ 1 & 3\end{array}\right]$

And inverse of the matrix $X$ is $X^{-1}=\left[\begin{array}{cc}-\frac{3}{4} & \frac{1}{4} \\ \frac{1}{4} & \frac{1}{4}\end{array}\right]$

The eigenvalue matrix is $\Lambda=\left[\begin{array}{ll}0 & 0 \\ 0 & 4\end{array}\right]$

Therefore, the matrix $A$ can be written as

$$
A=\left[\begin{array}{cc}
-1 & 1 \\
1 & 3
\end{array}\right]\left[\begin{array}{ll}
0 & 0 \\
0 & 4
\end{array}\right]\left[\begin{array}{cc}
-\frac{3}{4} & \frac{1}{4} \\
\frac{1}{4} & \frac{1}{4}
\end{array}\right]
$$

(b)

Consider that $A=X \Lambda X^{-1}$

The objective is to find that $A^{3}$ and $A^{-1}$

Find the value of $A^{3}$

$$
\begin{aligned}
A^{3} & =\left(X \Lambda X^{-1}\right) \cdot\left(X \Lambda X^{-1}\right) \cdot\left(X \Lambda X^{-1}\right) \\
& =\left(X \Lambda^{3} X^{-1}\right)
\end{aligned}
$$

Find the value of $A^{-1}$

$$
\begin{aligned}
A^{-1} & =(X \Lambda X)^{-1} \\
& =X \Lambda^{-1} X
\end{aligned}
$$
%
\section*{Problem set 6.2.3}
\subsection{Solve}
Consider that a matrix $A=X \Lambda X^{-1}$

Here, $X$ is eigenvector matrix and $\Lambda$ is eigenvalue matrix.

The objective is to find that the eigenvalue matrix and the eigenvector matrix for the matrix $A+2 I$.

The matrix $A=X \Lambda X^{-1}$

Hence, $A X=X \Lambda$

Let $\lambda$ be an eigenvalue of the matrix $A$ with respect to the eigenvector $x$.

Therefore, $A x=\lambda x$

Find the eigenvalue for the matrix $A+2 I$ is as follows:

$$
\begin{aligned}
(A+2 I) X & =A X+2 I X \quad \text { since } A X=X \Lambda \\
& =X \Lambda+X 2 \\
& =X(\Lambda+2)
\end{aligned}
$$

Hence, the value $(A+2 I) X=X(\Lambda+2)$

Therefore, $\Lambda+2$ is the eigenvalue matrix for the matrix $A+2 I$

The eigenvector matrix for matrix $A$ and matrix $A+2 I$ is same.

From equation (1) obtain, $(A+2 I)=X(\Lambda+2) X^{-1}$

Here, $\Lambda+2$ is the eigenvalue matrix for the matrix $A+2 I$ and $X$ is eigenvector matrix.
%
\end{document}